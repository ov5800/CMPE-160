%***********************************************************************
% Lab Report Template
% R. W. Melton
% May 19, 2019
% August 24, 2020
% January 4, 2021
%***********************************************************************
%Modify the following macros to set document property values
%used for cover sheet (title page) and page header
\newcommand{\CourseNumber}{CMPE-160}
\newcommand{\CourseName}{DSD1}
\newcommand{\SemesterName}{Spring 2021}
\newcommand{\SemesterCode}{2215}
\newcommand{\LabExNum}{<Lab Number>}
\newcommand{\LabExTitle}{<Practice>}
\newcommand{\StudentName}{Oliver Vinneras}
\newcommand{\DateSubmit}{<Date>}
\newcommand{\LabSection}{3}
\newcommand{\LabInstructor}{Mr. Lange}
\newcommand{\TAa}{Henry Bang}
\newcommand{\TAb}{Harrison Barnes}
\newcommand{\TAc}{Ian Chasse}
\newcommand{\TAd}{Shubhang Mehrotra}
\newcommand{\LectureSection}{1}
\newcommand{\LectureInstructor}{Mr. Cliver}
%End macros for document property values
%***********************************************************************
\title{Lab Ex. \LabExNum\ Report}
\author{\StudentName}
\date{\DateSubmit}
\makeatletter %make \title, \author, and \date availabile with \@
\newcommand{\FontSize}{12}
\newcommand{\FontUnit}{pt}
\newcommand{\HeadSize}{\dimexpr \FontSize\FontUnit + 2pt \relax}
\documentclass[\FontSize\FontUnit,letterpaper,oneside]{article}
\usepackage[twoside=false,margin=1in]{geometry}
\usepackage[utf8]{inputenc}
\usepackage[USenglish]{babel}
\usepackage{graphicx}
\usepackage[normalem]{ulem}
\usepackage{newtxtext, newtxmath}
%If newtx package had to be installed in multiuser environment
%regular user may have to run updmap command to avoid following error
%FATAL:  ``PK font ts1-qtmr could not be created.'' in miktex-makepk
%Alternatively, uncomment the following line
%\pdfmapfile{=pdftex35.map
\usepackage{booktabs}
\usepackage{enumitem}
\usepackage{nameref}
\usepackage[pdfborder={0 0 0},plainpages=false,pdfpagelabels]{hyperref}
%If hyperref generates errors on first build, rebuild.           
\hypersetup{pdfauthor={\@author},
            pdftitle={\@title},
            pdfsubject={\CourseNumber\ \SemesterCode},
            %pdfkeywords={},
            %pdfproducer={Latex with hyperref, or other system},
            %pdfcreator={pdflatex, or other tool}
            urlcolor=none}
\setlength{\topsep}{\z@}
\setlength{\partopsep}{\z@}
\setlength{\itemsep}{\z@}
\setlength{\parindent}{\z@}
\setlength{\parskip}{\FontSize\FontUnit plus 2pt minus 1pt}
\setlength{\baselineskip}{\dimexpr \FontSize\FontUnit + 2pt \relax}
\renewcommand \baselinestretch{1}
\makeatletter
  \renewcommand \section{
    \@startsection{section}{1}{\z@}
      %Before 2 lines, accounting for normal \parskip
      {\dimexpr \FontSize\FontUnit * 2 - \parskip \relax plus 0pt minus 0pt}
      %After 1 line, accounting for normal \parskip
      {0.1pt plus 2pt minus 1pt} %nonzero amount to get normal \parskip
      {\normalfont\normalsize\bfseries}}
  \renewcommand \subsection{
    \@startsection{paragraph}{2}{\z@}
      %Before 1 lines, accounting for normal \parskip
      {0.1pt plus 2pt minus 1pt}
      %After 0.5 em on same line as heading
      {-0.5em}
      {\normalfont\normalsize\bfseries}}
  \renewcommand \subsubsection{
    \@startsection{paragraph}{3}{\z@}
      %Before 1 line, accounting for normal \parskip
      {0.1pt plus 2pt minus 1pt}
      %After 0.5 em on same line as heading
      {-0.5em}
      {\normalfont\normalsize\uline}}
  \renewcommand \paragraph{
      \@startsection{paragraph}{4}{\z@}
      %Before 1 line, accounting for normal \parskip
      {0.1pt plus 2pt minus 1pt}
      %After 0.5 em on same line as heading
      {-0.5em}
      {\normalfont\normalsize}}
\makeatother
\pagenumbering{arabic}
\headheight=\HeadSize
\usepackage{fancyhdr}
\renewcommand{\headrulewidth}{0pt}
\renewcommand{\footrulewidth}{0pt}
\makeatletter %make \title, \author, and \date availabile with \@
\pagestyle{fancy}
\fancyhead{} %clear all header fields
\fancyhead[L]{\small \CourseNumber\ \SemesterCode\ \@author:  \@title}
\fancyhead[R]{\small Page \thepage\ of \pageref*{LastPage}}
\fancyfoot{} %clear all footer fields
\fancypagestyle{plain}{
  \renewcommand{\headrulewidth}{0pt}
  \renewcommand{\footrulewidth}{0pt}
  \fancyhf{} %clear header and footer fields
  \fancyfoot[C]{\small \CourseNumber\ \SemesterCode\ \@author:  \@title: 
    Page \thepage\ of \pageref*{LastPage}}
}
%May require second build to get correct page numbers.           
\begin{document}
\raggedbottom
\widowpenalties 1 10000
\lefthyphenmin=4
\righthyphenmin=4
\setlist{nolistsep}
%***********************************************************************
%Title page is automatically generated from macros at top of file
\pagenumbering{roman}
\begin{titlepage}
  %No space before paragraph at top of page
  %\vspace{\dimexpr-2\parsep-2\parskip\relax}
  %1.5 in before center (list) at top of page
  \vspace*{\dimexpr 1.5in - \topsep - \partopsep - \topskip - \parskip \relax}
  \begin{center}
    \textbf{\large\CourseNumber\ \CourseName\linebreak
      \linebreak
      Laboratory Exercise \LabExNum\linebreak
      \linebreak
      \LabExTitle}
  \end{center}
  \vspace*{\dimexpr 1.5in - \topsep - \partopsep - \topskip \relax}
  \par By submitting this report, I attest that its contents are wholly
    my individual writing about this exercise and that they reflect
    the submitted code.  I further acknowledge that permitted
    collaboration for this exercise consists only of discussions of
    concepts with course staff and fellow students.  Other than code
    provided by the instructor for this exercise, all code was
    developed by me.
  \null
  \vspace*{4\parskip}
  \hspace*{3.25in}\begin{tabular}[t]
    {@{\hskip0pt}r    %Specification
     @{\hskip1em}l    %Value
     @{\hskip0pt}}
    \toprule[1pt]
    \multicolumn{2}{l}{\StudentName}\\
    \multicolumn{2}{l}{\DateSubmit}\\
    \\
    Lab Section:&\LabSection\\
    Instructor:&\LabInstructor\\
    TA:&\TAa\\
    &\TAb\\
    &\TAc\\
    &\TAd\\
    \\
    Lecture Section:&\LectureSection\\
    Lecture Instructor:&\LectureInstructor
  \end{tabular}
\end{titlepage}
\pagenumbering{arabic}
\thispagestyle{plain}
%***********************************************************************
%Report body begins here
\section*{Abstract}
  [\textit{Verify in the ``Lab Report'' section of the lab assignment
    whether this section is required for this exercise.}]
\par This is where you write the Abstract
\section*{Design Methodology}
[\textit{Verify in the ``Lab Report'' section of the lab assignment
  whether this section is required for this exercise.}]
\par The design methodology presents much of the theory behind the lab
  exercise, which was incorporated in the program.  Figures, equations,
  and tables are often very effective in this presentation.  Any figures
  or tables should be integrated within the report and must be properly
  labeled, referenced, and explained in text, (e.g., ``the final design
  of the timer game is depicted in Figure 5'').  The design methodology
  should stand on its own objective technical merit without any
  reference to class or lab, (e.g., not any of the following: 
  ``\sout{the design methodology was based on the lab assignment},''
  ``\sout{the specifications given by the instructor were followed},''
  etc.).
\section*{Procedure}
[\textit{Verify in the ``Lab Report'' section of the lab assignment
  whether this section is required for this exercise.}]
\par The procedure tells what was done and how it was done in a
  chronological, narrative style--in contrast to a lab assignment’s
  procedure section, which is typically written in imperative style,
  (i.e., list of instructions).  The report’s procedure should tell in
  the writer’s own words what was actually done in lab.  From reading
  the procedure, the reader should be able to perform the exercise to
  get the same results as the writer presents.
\section*{Results}
[\textit{Verify in the ``Lab Report'' section of the lab assignment
  whether this section is required for this exercise.}]
\par The results section provides the reader with all results that came from
  completing the lab exercise, (e.g., any register values or memory
  contents that were generated).  For most exercises, the results can be
  incorporated in a figure from a screen capture or from a log file. 
  As in the rest of the report, every figure, table, etc. must be
  properly labeled, introduced, and discussed in text.
\section*{Conclusion}
[\textit{Verify in the ``Lab Report'' section of the lab assignment
  whether this section is required for this exercise.}]
\par The conclusion section provides the reader with insights that came from
  completing the lab exercise.  It describes the lessons learned in the
  exercise and analyzes the results.  It also notes and explains any
  deviations from the assigned procedure.  It concludes by describing
  general concepts applied to achieve the results and by making
  statements that extend beyond the scope of the exercise to the general
  case, (e.g., where the concepts or results could be applied outside of
  the lab exercise).
\label{LastPage}
\end{document}